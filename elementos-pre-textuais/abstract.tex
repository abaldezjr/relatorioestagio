The present graduation work presents the project of a system of Cartesian coordinates 
for positioning measuring instruments located in the wind tunnel of the laboratory 
of the Federal University of Rio Grande. The design of the project started from the 
need for precision and repeatability in displacement and positioning of measuring 
instruments that characterize the speed and / or pressure field in the test section 
of the streamline. For the development of this system, free hardware platforms were 
used, such as arduino, aiming at low cost and easy reproduction. The system consists 
of an application that will receive the coordinate to where the table should move. 
The project can be divided into three parts: mechanical, in which it involves the 
disposition of mechanical components; electrical, on which the electrical circuits 
were designed; and the programming, in which the application's command system was 
developed and the cartesian table. Therefore, for the composition of the electronic 
system an Arduino Uno controller board, power drivers, stepper motors, optocouplers 
and encoders were used. For the mechanical system, axes, shafts, bearings, spindles 
and aluminum structure (base, tower) were used.

\keywords{Wind tunnel. Pitot tube. Arduino. Cartesian table.}