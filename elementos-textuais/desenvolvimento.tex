\chapter{Atividades desenvolvidas no estágio}\label{ch:desenvolvimento}

\section{Descrição geral}\label{sec:descgeral}

O estágio foi realizado no setor de Retifica de motores e Manutenção 
na Eletrodiesel Krios - Retifica de Motores.

\section{Lista das atividades}\label{sec:cronograma}

O estágio foi constituido em 30~h semanais, 6~h ao dia, totalizando
180~h. Neste horário várias atividades foram efetuadas, entre elas:

\begin{enumerate}
    \item Acompanhamento no diagnóstico de defeitos de motores de combustão interna;
    \item Serviços de fresa;
    \item Plaina;
    \item Retífica de cilindros;
    \item Auxílio na desmontagem e montagem de motores;
    \item Acompanhamento em manutenção naval;
\end{enumerate}

\section{Orientação e Supervisão}\label{sec:orientacao}

A orientação foi efetuada pelo Professor Doutor Marcio Ulguim Oliveira e a supervisão foi 
efetuada pelo proprietário da empresa Jorge Volnei Rios Kwecko.

\section{Aplicação de conhecimento}\label{sec:aplicacao}

No estágio realizado se colocou em prática diversos conceitos obtidos durante a 
graduação em Engenharia Mecânica, sendo assim de suma importância para a carreira.

\section{Dificuldades Encontradas}\label{sec:dificuldades}

A maior dificuldade, foi relacionado a proteção e afastamento social devido a Pandemia
do Covid-19. 

\section{Relacionamento Humano}\label{sec:relacionamento}

Os funcionários efetivos da empresa, assim como o supervisor foram
essenciais para o aprendizado, sempre dispostos a colaborar, explicando 
como funcionava a realização de cada tarefa.
